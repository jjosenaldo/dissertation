\section{Propositional Dynamic Logic}
Propositional Dynamic Logic (PDL), as its name suggests, is the propositional counterpart of dynamic logic. The latter has its origins in 
%TODO: search for the appropriate reference
[], and it was created with the goal of reasoning about imperative computer programs.

\subsection{Language}

The language of PDL has two kinds of expressions: formulas and programs. The former will be usually represented by $\phi$, $\psi$, $\gamma$, \etc, and the variables $\pi_1, \pi_2, \etc$ range over programs.

\begin{definition}
The set $\Phi$ of formulas consists of a countable number of propositional variables $p_1, p_2,\dots$ plus the following grammar:
%
\begin{align*}
    \phi & \bnfdef \bot \mid \lnot\phi \mid \phi \land \psi \mid \phi \to \psi
    \mid \phi \liff  \psi\mid \necess{\pi}{\phi} \mid \possib{\pi}{\phi}.
\end{align*}
We can introduce the remaining  connective $\lor$ via the abbreviation $\phi \lor \psi \assign (\phi \to \bot)\to \psi$. Note that the program-free formulas are precisely the formulas of Classical Propositional Logic (CPL). Thus, PDL is an extension of CPL.
\end{definition}

An informal account of $\necess \pi \phi$ is that the formula $\phi$ is \emph{necessarily} true after the execution of the program $\pi$. As for $\possib\pi\phi$, it intuitively means that $\phi$ may be true --- or, rather, is \emph{possibly} true --- after $\pi$ finishes its execution. 
It turns out that necessity and possibility are dual notions: for every $\phi$ and $\pi$, it is the case that $\possib{\pi}{\phi} = \lnot \necess\pi{\lnot \phi}$ and $\necess{\pi}{\phi} = \lnot \possib\pi{\lnot \phi}$. This shall be verified later, when we introduce the rules of the logic.

The operators $\necess{\ }{}$ and $\possib{\ }{}$ are also often referred to as \emph{box} and \emph{diamond}, respectively. The resemblance with the modalities of modal logic is not a casualty: PDL is a multimodal logic. Each program $\pi$ gives rise to the modalities $\necess{\pi}{}$ and $\possib{\pi}{}$.

\begin{definition}
The set $\Pi$ of  PDL programs is composed of countable many atomic programs $a_1$, $a_2$,~\etc, and the following grammar:
%
\begin{align*}
    \pi & \bnfdef \pi_1\comp \pi_2 \mid \pi_1\choice \pi_2 \mid \iter \pi_1 \mid \test \phi
\end{align*}
\end{definition}
%
The program $\pi_1\comp \pi_2$ is the sequential composition of $\pi_1$ and $\pi_2$, i.e., the output of $\pi_1$ is used as the input of $\pi_2$. As for $\pi_1 \choice \pi_2$, it either behaves as $\pi_1$ or $\pi_2$; such choice being nondeterministic. The idea of repetition of repetition is captured by $\iter \pi_1$, which is read as executing $\pi_1$ a nondeterministic number of times. Lastly, the command $\phi?$ halts the execution of the main program if $\phi$ is false and goes to the next instruction otherwise.

\subsection{Deductive System}
One way of defining a deductive system for PDL is by a Hilbert-style axiom scheme and some deduction rules. A slight modification of the system given in \cite{harel2001dl} is presented here. We have chosen a natural deduction set of rules for instances of CPL formulas because it is more readable.

\begin{definition}
The PDL axioms are:
\begin{enumerate}[label=({A\arabic*})]
    \item $\necess{\alpha}{\paren{{\phi \to \psi}}} \to \paren{\necess\alpha\phi \to \necess\alpha\psi}$,
    \item $\necess\alpha{\paren{\phi\land\psi}} \liff \necess\alpha\phi \land \necess\alpha\psi$,
    \item $\necess{\alpha\choice\beta}\phi \liff \necess\phi\land\necess\beta\phi$,
    \item $\necess{\alpha\comp\beta}\liff\necess\alpha{\necess\beta\phi}$,
    \item $\necess{\test\psi}\phi\liff(\psi\to\phi)$,
    \item $\phi\land\necess{\alpha}{\necess{\iter\alpha}\phi} \liff \necess{\iter \alpha}\phi$,
    \item \label{item:pdlaxiom_induction} $\phi\land\necess{\iter\alpha}{\paren{\phi\to\necess\alpha\phi}}\to\necess{\iter\alpha}\phi$, or the \emph{induction} axiom.
\end{enumerate}
As for deduction rules, we consider the instances of natural deduction rules of CPL plus the GEN rule:
%
\begin{displaymath}
\prftree[r]{GEN}{\phi}{\necess{\alpha}{\phi}}
\end{displaymath}
%
In the GEN rule, $\phi$ is required to be a theorem of PDL.
\end{definition}

In order to see the deductive system in action, let us now derive some rules and theorems.

\begin{example}[Admissibiblity of MON]
The MON rule, whose name stands for \emph{monotonicity}, is the following scheme:
\begin{displaymath}
\prftree[r]{MON}{\phi \to \psi}{\necess{\alpha}{\phi} \to \necess{\alpha}{\psi}}
\end{displaymath}
%
Its admissibility can be shown by means of the following deduction tree:
%
\begin{displaymath}
\prftree[r]{MP}{
\prfbyaxiom{A1}{
\necess{\alpha}{(\phi \to \psi)} \to (\necess{\alpha}{\phi} \to \necess{\alpha}{\psi})
}
}{
\prftree[r]{NEC}{\phi \to \psi}{
\necess{\alpha}{(\phi \to \psi)}
}
}{
\necess{\alpha}{\phi} \to \necess{\alpha}{\psi}
}
\end{displaymath}
\end{example}

\begin{example}[Admissibility of LI]
Other admissible rule in PDL is the LI rule --- its name is a short for \emph{loop invariant} ---:
%
\begin{displaymath}
\prftree[r]{LI}{\phi \to \necess{\alpha}{\phi}}{
\phi \to \necess{\iter \alpha}{\phi}
}
\end{displaymath}
%
Here is a proof that the aforementioned rule is indeed admissible:
% I guess that this redness is just a bug in the syntax highlighter of Overleaf, as no warnings or compilation errors are thrown.
\begin{displaymath}
\prftree[r]{$\to\text{I}_{\prfref<assum:A>}$}{
\prfboundedstyle=1
\prftree[r]{$\to\text{E}$}{
\prfbyaxiom{A7}{\phi \land \necess{\iter \alpha}{\paren{\phi \to \necess{\alpha}{\phi}}} \to \necess{\iter \alpha}{\phi}}
}{
\prftree[r]{$\land\text{I}$}{
\prfboundedassumption<assum:A>{\phi }
}{
\prftree[r]{NEC}{
\prfassumption{\phi \to \necess{\alpha}{\phi}}
}{
\necess{\iter \alpha}{\paren{\phi \to \necess{\alpha}{\phi}}}
}
}{
\phi \land \necess{\iter \alpha}{\paren{\phi \to \necess{\alpha}{\phi}}}
}
}{
\necess{\iter \alpha}{\phi}
}
}{
\phi \to \necess{\iter \alpha}{\phi}
}
\end{displaymath}
\end{example}

\begin{example}
The formula $\necess{\iter \alpha}{\phi} \to \necess{\alpha}{\phi}$ is a theorem of PDL. This fact can be proved using the MON rule:
%
\begin{displaymath}
\begin{prfenv}
\prftree[r]{$\to\text{I}_{\prfref<assum:x>}$}{
\prfboundedstyle=1
\prftree[r]{$\to\text{E}$}{
\prftree[r]{MON}{
\prftree[r]{$\to\text{I}_{\prfref<assum:y>}$}{
\prftree[r]{$\land\text{E}$}{
\prftree[r]{$\liff\text{E}$}{
\prfbyaxiom{A6}{\phi \land \necess{\alpha}{\necess{\iter \alpha}{\phi}} \liff \necess{\iter \alpha}{\phi}}
}{
\prfboundedassumption<assum:y>{\necess{\iter \alpha}{\phi}}
}{
\phi \land \necess{\iter \alpha}{\necess{\iter \alpha}{\phi}}
}
}{
\phi
}
}{
\necess{\iter \alpha}{\phi} \to \phi
}
}{
\necess{\alpha}{\necess{\iter \alpha}{\phi}} \to \necess{\alpha}{\phi}
}
}{
\prftree[r]{$\to\text{E}$}{
\prftree[r]{$\to\text{I}_{\prfref<assum:z>}$}{
\prftree[r]{$\land\text{E}$}{
\prftree[r]{$\liff\text{E}$}{
\prfbyaxiom{A6}{\phi \land \necess{\alpha}{\necess{\iter \alpha}{\phi}} \liff \necess{\iter \alpha}{\phi}}
}{
\prfboundedassumption<assum:z>{\necess{\iter \alpha}{\phi}}
}{
\phi \land \necess{\alpha}{\necess{\iter \alpha}{\phi}}
}
}{
\necess{\alpha}{\necess{\iter \alpha}{\phi}}
}
}{
\necess{\iter \alpha}{\phi} \to \necess{\alpha}{\necess{\iter \alpha}{\phi}}
}
}{
\prfboundedassumption<assum:x>{\necess{\iter \alpha}{\phi}}
}{
\necess{\alpha}{\necess{\iter \alpha}{\phi}}
}
}{
\necess{\alpha}{\phi}
}
}{
\necess{\iter \alpha}{\phi} \to \necess{\alpha}{\phi}
}
\end{prfenv}
\end{displaymath}
\end{example}

\subsubsection{Semantics}

% what was DL created for
% two types of expressions: formulas and programs
% syntax
% modality
% kripke semantics