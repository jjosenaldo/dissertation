\PassOptionsToPackage{main=english, brazilian}{babel}

\documentclass[
	% -- opções da classe memoir --
	12pt,				% tamanho da fonte
	openright,			% capítulos começam em pág ímpar (insere página vazia caso preciso)
	oneside,			% para impressão em recto e verso. Oposto a oneside
	a4paper			% tamanho do papel
	]{abntex2}
% ---
% Pacotes básicos 
% ---
\usepackage{lmodern} % usa a fonte Latin Modern		
\usepackage[T1]{fontenc} % seleção de codigos de fonte.
\usepackage[utf8]{inputenc} % codificação do documento (conversão automática dos acentos)
\usepackage{indentfirst}% indenta o primeiro parágrafo de cada seção.
\usepackage{color} % controle das cores
\usepackage{graphicx} % unclusão de gráficos
\usepackage{microtype} % para melhorias de justificação
\usepackage[brazilian,hyperpageref]{backref} % páginas com as citações na bibl
\usepackage[alf]{abntex2cite} % citações padrão ABNT
\usepackage{lipsum} % geração de dummy text
\usepackage{ppgmae_config} % configurações específicas do modelo do PPgMAE (no geral, esse arquivo não deve ser modificado)

\usepackage{amssymb,amsmath} % align*
\usepackage{amsthm} % \theoremstyle
\usepackage{prftree}
\usepackage{mycommands}

\begin{document}
% Retira espaço extra obsoleto entre as frases.
\frenchspacing 

% ----------------------------------------------------------
% ELEMENTOS PRÉ-TEXTUAIS
% ----------------------------------------------------------
\pretextual

% Capa
% \imprimircapa

% Folha de rosto
% (o * indica que haverá a ficha bibliográfica)
% \imprimirfolhaderosto*

% Ficha bibliográfica
% % Isto é um exemplo de Ficha Catalográfica, ou ``Dados internacionais de
% catalogação-na-publicação''. Você pode utilizar este modelo como referência. 
% Porém, provavelmente a biblioteca da sua universidade lhe fornecerá um PDF
% com a ficha catalográfica definitiva após a defesa do trabalho. Quando estiver
% com o documento, salve-o como PDF no diretório do seu projeto e substitua todo
% o conteúdo de implementação deste arquivo pelo comando abaixo:
%
% \begin{fichacatalografica}
%     \includepdf{fig_ficha_catalografica.pdf}
% \end{fichacatalografica}

\begin{fichacatalografica}
	\sffamily
	\vspace*{\fill}					% Posição vertical
	\begin{center}					% Minipage Centralizado
	\fbox{\begin{minipage}[c][8cm]{13.5cm}		% Largura
	\small
	\imprimirautor
	%Sobrenome, Nome do autor
	
	\hspace{0.5cm} \imprimirtitulo  / \imprimirautor. --
	\imprimirlocal, \imprimirdata-
	
	\hspace{0.5cm} \thelastpage p. : il. (algumas color.) ; 30 cm.\\
	
	\hspace{0.5cm} \imprimirorientadorRotulo~\imprimirorientador\\
	
	\hspace{0.5cm}
	\parbox[t]{\textwidth}{\imprimirtipotrabalho~--~\imprimirinstituicao,
	\imprimirdata.}\\
	
	\hspace{0.5cm}
		1. Palavra-chave1.
		2. Palavra-chave2.
		2. Palavra-chave3.
		I. Orientador.
		II. Universidade xxx.
		III. Faculdade de xxx.
		IV. Título.			
	\end{minipage}}
	\end{center}
\end{fichacatalografica}

% Erratas
% \begin{errata}
Elemento opcional da \citeonline[4.2.1.2]{NBR14724:2011}. Exemplo:

\vspace{\onelineskip}

FERRIGNO, C. R. A. \textbf{Tratamento de neoplasias ósseas apendiculares com
reimplantação de enxerto ósseo autólogo autoclavado associado ao plasma
rico em plaquetas}: estudo crítico na cirurgia de preservação de membro em
cães. 2011. 128 f. Tese (Livre-Docência) - Faculdade de Medicina Veterinária e
Zootecnia, Universidade de São Paulo, São Paulo, 2011.

\begin{table}[htb]
\center
\footnotesize
\begin{tabular}{|p{1.4cm}|p{1cm}|p{3cm}|p{3cm}|}
  \hline
   \textbf{Folha} & \textbf{Linha}  & \textbf{Onde se lê}  & \textbf{Leia-se}  \\
    \hline
    1 & 10 & auto-conclavo & autoconclavo\\
   \hline
\end{tabular}
\end{table}

\end{errata}

% Folha de aprovação
% % Isto é um exemplo de Folha de aprovação, elemento obrigatório da NBR
% 14724/2011 (seção 4.2.1.3). Você pode utilizar este modelo até a aprovação
% do trabalho. Após isso, substitua todo o conteúdo deste arquivo por uma
% imagem da página assinada pela banca com o comando abaixo:
%
% \begin{folhadeaprovacao}
% \includepdf{folhadeaprovacao_final.pdf}
% \end{folhadeaprovacao}
%
\begin{folhadeaprovacao}
\noindent\introducaofolhadeaprovacao

\assinatura{\textbf{\imprimirorientador} \\ \imprimirorientadorRotulo \\ Departamento de Matemática \\ Universidade Federal do Rio Grande do Norte} 
\assinatura{\textbf{Prof. Dr. Carlos Alberto Olarte Vega} \\ Coorientador \\ Escola de Ciências e Tecnologia \\ Universidade Federal do Rio Grande do Norte}
\assinatura{\textbf{Prof. Dr. Mario Roberto Folhadela Benevides} \\ Coorientador \\ Departamento de Ciência da Computação \\ Universidade Federal do Rio de Janeiro}
\vfill
\begin{center}
\vspace*{0.5cm}
{\large\imprimirlocal}
\par
{\large\imprimirdata}
\vspace*{1cm}
\end{center}
  
\end{folhadeaprovacao}

% Dedicatória
% \begin{dedicatoria}
   \vspace*{\fill}
   \centering
   \noindent
   \textit{Homenagem que o autor presta a uma ou mais pessoas.} \vspace*{\fill}
\end{dedicatoria}

% Agradecimentos
% \begin{agradecimentos}
Agradecimentos dirigidos àqueles que contribuíram de maneira relevante à elaboração do trabalho, sejam eles pessoas ou mesmo organizações.
\end{agradecimentos}


% Epígrafe
% \begin{epigrafe}
    \vspace*{\fill}
	\begin{flushright}
		\textit{``Contrariwise, if it was so, it might be;\\ and if it were so, it would be;\\ but as it isn't, it ain't. That's logic.''\\
		(Lewis Carroll)}
	\end{flushright}
\end{epigrafe}

% Resumo em português
% \begin{resumo}
O resumo deve apresentar de forma concisa os pontos relevantes de um texto, fornecendo uma visão rápida e clara do conteúdo e das conclusões do trabalho. O texto, redigido na forma impessoal do verbo, é constituído de uma seqüência de frases concisas e objetivas e não de uma simples enumeração de tópicos, não ultrapassando 500 palavras, seguido, logo
abaixo, das palavras representativas do conteúdo do trabalho, isto é, palavras-chave e/ou descritores. Por fim, deve-se evitar, na redação do resumo, o uso de parágrafos (em geral resumos são escritos em parágrafo único), bem como de fórmulas, equações, diagramas e símbolos, optando-se, quando necessário,

\textbf{Palavras-chave}: Palavra-chave 1. Palavra-chave 2. Palavra-chave 3.
\end{resumo}


% Resumo em inglês
% \begin{resumo}[Abstract]
\begin{otherlanguage*}{english}
    O resumo em língua estrangeira (em inglês Abstract, em espanhol Resumen, em francês Résumé) é uma versão do resumo escrito na língua vernácula para idioma de divulgação internacional. Ele deve apresentar as mesmas características do anterior (incluindo as mesmas palavras, isto é, seu conteúdo não deve diferir do resumo anterior), bem como ser seguido das palavras representativas do conteúdo do trabalho, isto é, palavras-chave
    e/ou descritores, na língua estrangeira. Embora a especificação abaixo considere o inglês como língua estrangeira (o mais comum), não fica impedido a adoção de outras línguas (a exemplo de espanhol ou francês) para redação do resumo em língua estrangeira.
    \vspace{\onelineskip}
    
    \noindent 
    \textbf{Keywords}: Keyword 1. Keyword 2. Keyword 3.
 \end{otherlanguage*}
\end{resumo}

% Lista de ilustrações
% \inserirlistailustracoes

% Lista de quadros
% \inserirlistaquadros

% Lista de tabelas
% \inserirlistatabelas

% Lista de abreviaturas e siglas
\begin{siglas}
  \item[CPL] Classical Propositional Logic
  \item[DL] Dynamic Logic
  \item[PDL] Propositional Dynamic Logic
\end{siglas}

% Lista de símbolos
% \begin{simbolos}
  \item[$ \Gamma $] Letra grega Gama
  \item[$ \Lambda $] Lambda
  \item[$ \zeta $] Letra grega minúscula zeta
  \item[$ \in $] Pertence
\end{simbolos}

% Sumário
\inserirsumario

% ----------------------------------------------------------
% ELEMENTOS TEXTUAIS
% ----------------------------------------------------------
\textual

\chapter{Introduction}

% STRUCTURE
% intro - logics and computation
%    concurrent systems
%    semantics of programming languages
%    process algebra
%    artificial intelligence
% PDL
% linear logic
% goal of this work
% structure of the dissertation

The applications of logics in computer science range over diverse fields and concepts. Some examples are artificial intelligence \cite{hu2016neuralNetworks}, concurrency \cite{caires2003spatial}, parallel systems \cite{iliano2012ccsll} and semantics of programming languages \cite{ellison2012executable}. This variety of applications is not incidental; the very nature of logic as a means of modeling how consequences follow from a set of premises is akin to that of computation.

A particular logic which was specifically created with computational goals in mind is Dynamic Logic (DL).
In its original version, DL is a modal logic containing modalities.
\chapter{Preliminaries}
This chapter will introduce the two fundamental building blocks of this thesis; namely, propositional dynamic logic and linear logic.

\section{Propositional Dynamic Logic}
Propositional Dynamic Logic (PDL), as its name suggests, is the propositional counterpart of dynamic logic. The latter has its origins in 
%TODO: search for the appropriate reference
[], and it was created with the goal of reasoning about imperative computer programs.

\subsection{Language}

The language of PDL has two kinds of expressions: formulas and programs. The former will be usually represented by $\phi$, $\psi$, $\gamma$, \etc, and the variables $\pi_1, \pi_2, \etc$ range over programs.

\begin{definition}
The set $\Phi$ of formulas consists of a countable number of propositional variables $p_1, p_2,\dots$ plus the following grammar:
%
\begin{align*}
    \phi & \bnfdef \bot \mid \lnot\phi \mid \phi \land \psi \mid \phi \to \psi
    \mid \phi \liff  \psi\mid \necess{\pi}{\phi} \mid \possib{\pi}{\phi}.
\end{align*}
We can introduce the remaining  connective $\lor$ via the abbreviation $\phi \lor \psi \assign (\phi \to \bot)\to \psi$. Note that the program-free formulas are precisely the formulas of Classical Propositional Logic (CPL). Thus, PDL is an extension of CPL.
\end{definition}

An informal account of $\necess \pi \phi$ is that the formula $\phi$ is \emph{necessarily} true after the execution of the program $\pi$. As for $\possib\pi\phi$, it intuitively means that $\phi$ may be true --- or, rather, is \emph{possibly} true --- after $\pi$ finishes its execution. 
It turns out that necessity and possibility are dual notions: for every $\phi$ and $\pi$, it is the case that $\possib{\pi}{\phi} = \lnot \necess\pi{\lnot \phi}$ and $\necess{\pi}{\phi} = \lnot \possib\pi{\lnot \phi}$. This shall be verified later, when we introduce the rules of the logic.

The operators $\necess{\ }{}$ and $\possib{\ }{}$ are also often referred to as \emph{box} and \emph{diamond}, respectively. The resemblance with the modalities of modal logic is not a casualty: PDL is a multimodal logic. Each program $\pi$ gives rise to the modalities $\necess{\pi}{}$ and $\possib{\pi}{}$.

\begin{definition}
The set $\Pi$ of  PDL programs is composed of countable many atomic programs $a_1$, $a_2$,~\etc, and the following grammar:
%
\begin{align*}
    \pi & \bnfdef \pi_1\comp \pi_2 \mid \pi_1\choice \pi_2 \mid \iter \pi_1 \mid \test \phi
\end{align*}
\end{definition}
%
The program $\pi_1\comp \pi_2$ is the sequential composition of $\pi_1$ and $\pi_2$, i.e., the output of $\pi_1$ is used as the input of $\pi_2$. As for $\pi_1 \choice \pi_2$, it either behaves as $\pi_1$ or $\pi_2$; such choice being nondeterministic. The idea of repetition of repetition is captured by $\iter \pi_1$, which is read as executing $\pi_1$ a nondeterministic number of times. Lastly, the command $\phi?$ halts the execution of the main program if $\phi$ is false and goes to the next instruction otherwise.

\subsection{Deductive System}
One way of defining a deductive system for PDL is by a Hilbert-style axiom scheme and some deduction rules. A slight modification of the system given in \cite{harel2001dl} is presented here. We have chosen a natural deduction set of rules for instances of CPL formulas because it is more readable.

\begin{definition}
The PDL axioms are:
\begin{enumerate}[label=({A\arabic*})]
    \item $\necess{\alpha}{\paren{{\phi \to \psi}}} \to \paren{\necess\alpha\phi \to \necess\alpha\psi}$,
    \item $\necess\alpha{\paren{\phi\land\psi}} \liff \necess\alpha\phi \land \necess\alpha\psi$,
    \item $\necess{\alpha\choice\beta}\phi \liff \necess\phi\land\necess\beta\phi$,
    \item $\necess{\alpha\comp\beta}\liff\necess\alpha{\necess\beta\phi}$,
    \item $\necess{\test\psi}\phi\liff(\psi\to\phi)$,
    \item $\phi\land\necess{\alpha}{\necess{\iter\alpha}\phi} \liff \necess{\iter \alpha}\phi$,
    \item \label{item:pdlaxiom_induction} $\phi\land\necess{\iter\alpha}{\paren{\phi\to\necess\alpha\phi}}\to\necess{\iter\alpha}\phi$, or the \emph{induction} axiom.
\end{enumerate}
As for deduction rules, we consider the instances of natural deduction rules of CPL plus the GEN rule:
%
\begin{displaymath}
\prftree[r]{GEN}{\phi}{\necess{\alpha}{\phi}}
\end{displaymath}
%
In the GEN rule, $\phi$ is required to be a theorem of PDL.
\end{definition}

In order to see the deductive system in action, let us now derive some rules and theorems.

\begin{example}[Admissibiblity of MON]
The MON rule, whose name stands for \emph{monotonicity}, is the following scheme:
\begin{displaymath}
\prftree[r]{MON}{\phi \to \psi}{\necess{\alpha}{\phi} \to \necess{\alpha}{\psi}}
\end{displaymath}
%
Its admissibility can be shown by means of the following deduction tree:
%
\begin{displaymath}
\prftree[r]{MP}{
\prfbyaxiom{A1}{
\necess{\alpha}{(\phi \to \psi)} \to (\necess{\alpha}{\phi} \to \necess{\alpha}{\psi})
}
}{
\prftree[r]{NEC}{\phi \to \psi}{
\necess{\alpha}{(\phi \to \psi)}
}
}{
\necess{\alpha}{\phi} \to \necess{\alpha}{\psi}
}
\end{displaymath}
\end{example}

\begin{example}[Admissibility of LI]
Other admissible rule in PDL is the LI rule --- its name is a short for \emph{loop invariant} ---:
%
\begin{displaymath}
\prftree[r]{LI}{\phi \to \necess{\alpha}{\phi}}{
\phi \to \necess{\iter \alpha}{\phi}
}
\end{displaymath}
%
Here is a proof that the aforementioned rule is indeed admissible:
% I guess that this redness is just a bug in the syntax highlighter of Overleaf, as no warnings or compilation errors are thrown.
\begin{displaymath}
\prftree[r]{$\to\text{I}_{\prfref<assum:A>}$}{
\prfboundedstyle=1
\prftree[r]{$\to\text{E}$}{
\prfbyaxiom{A7}{\phi \land \necess{\iter \alpha}{\paren{\phi \to \necess{\alpha}{\phi}}} \to \necess{\iter \alpha}{\phi}}
}{
\prftree[r]{$\land\text{I}$}{
\prfboundedassumption<assum:A>{\phi }
}{
\prftree[r]{NEC}{
\prfassumption{\phi \to \necess{\alpha}{\phi}}
}{
\necess{\iter \alpha}{\paren{\phi \to \necess{\alpha}{\phi}}}
}
}{
\phi \land \necess{\iter \alpha}{\paren{\phi \to \necess{\alpha}{\phi}}}
}
}{
\necess{\iter \alpha}{\phi}
}
}{
\phi \to \necess{\iter \alpha}{\phi}
}
\end{displaymath}
\end{example}

\begin{example}
The formula $\necess{\iter \alpha}{\phi} \to \necess{\alpha}{\phi}$ is a theorem of PDL. This fact can be proved using the MON rule:
%
\begin{displaymath}
\begin{prfenv}
\prftree[r]{$\to\text{I}_{\prfref<assum:x>}$}{
\prfboundedstyle=1
\prftree[r]{$\to\text{E}$}{
\prftree[r]{MON}{
\prftree[r]{$\to\text{I}_{\prfref<assum:y>}$}{
\prftree[r]{$\land\text{E}$}{
\prftree[r]{$\liff\text{E}$}{
\prfbyaxiom{A6}{\phi \land \necess{\alpha}{\necess{\iter \alpha}{\phi}} \liff \necess{\iter \alpha}{\phi}}
}{
\prfboundedassumption<assum:y>{\necess{\iter \alpha}{\phi}}
}{
\phi \land \necess{\iter \alpha}{\necess{\iter \alpha}{\phi}}
}
}{
\phi
}
}{
\necess{\iter \alpha}{\phi} \to \phi
}
}{
\necess{\alpha}{\necess{\iter \alpha}{\phi}} \to \necess{\alpha}{\phi}
}
}{
\prftree[r]{$\to\text{E}$}{
\prftree[r]{$\to\text{I}_{\prfref<assum:z>}$}{
\prftree[r]{$\land\text{E}$}{
\prftree[r]{$\liff\text{E}$}{
\prfbyaxiom{A6}{\phi \land \necess{\alpha}{\necess{\iter \alpha}{\phi}} \liff \necess{\iter \alpha}{\phi}}
}{
\prfboundedassumption<assum:z>{\necess{\iter \alpha}{\phi}}
}{
\phi \land \necess{\alpha}{\necess{\iter \alpha}{\phi}}
}
}{
\necess{\alpha}{\necess{\iter \alpha}{\phi}}
}
}{
\necess{\iter \alpha}{\phi} \to \necess{\alpha}{\necess{\iter \alpha}{\phi}}
}
}{
\prfboundedassumption<assum:x>{\necess{\iter \alpha}{\phi}}
}{
\necess{\alpha}{\necess{\iter \alpha}{\phi}}
}
}{
\necess{\alpha}{\phi}
}
}{
\necess{\iter \alpha}{\phi} \to \necess{\alpha}{\phi}
}
\end{prfenv}
\end{displaymath}
\end{example}

\subsubsection{Semantics}

% what was DL created for
% two types of expressions: formulas and programs
% syntax
% modality
% kripke semantics

% ----------------------------------------------------------
% ELEMENTOS PÓS-TEXTUAIS
% ----------------------------------------------------------
\postextual

% Referências bibliográficas
\bibliography{referencias}

% Apêndices
% \begin{apendicesenv}

% Imprime uma página indicando o início dos apêndices
\partapendices

% ----------------------------------------------------------
\chapter{Quisque libero justo}
% ----------------------------------------------------------

\lipsum[50]

% ----------------------------------------------------------
\chapter{Nullam elementum urna vel imperdiet sodales elit ipsum pharetra ligula
ac pretium ante justo a nulla curabitur tristique arcu eu metus}
% ----------------------------------------------------------
\lipsum[55-57]

\end{apendicesenv}


% Anexos
% \begin{anexosenv}

% Imprime uma página indicando o início dos anexos
\partanexos

% ---
\chapter{Morbi ultrices rutrum lorem.}
% ---
\lipsum[30]

% ---
\chapter{Cras non urna sed feugiat cum sociis natoque penatibus et magnis dis
parturient montes nascetur ridiculus mus}
% ---

\lipsum[31]

% ---
\chapter{Fusce facilisis lacinia dui}
% ---

\lipsum[32]

\end{anexosenv}

\end{document}
